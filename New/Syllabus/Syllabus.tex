\documentclass[12pt]{article}

\usepackage{Physics,python,miller,indentfirst}
% style=verbose-trad1
% style=authortitle-icomp [default]
\usepackage[style=verbose-trad1]{biblatex} 
\pagestyle{fancy}
\pagenumbering{gobble}
\addbibresource{References.bib}

\begin{document}

\lhead{ASTR 101}
\rhead{Fall 2019}
%\chead{Maziar Saleh Ziabari}

\newcommand{\TODO}[1]{}

\br\vspace{0cm}\begin{center}\Huge Astronomy 101 Syllabus\end{center}\vspace{.5cm}

\noindent\begin{tabular}{>{\it}r l}
  Instructor:&Maziar Saleh Ziabari\\
  Email:&\href{mailto:helasraizam@unm.edu}{helasraizam@unm.edu}\\
  OH:&Saturdays 7-9 pm\\
  Classroom:&online\\
  Site:&\url{https://learn.unm.edu}, \url{https://helasraizam.github.io}
\end{tabular}
\vspace{.7cm}

Welcome to Astronomy 101!  Astronomy is the study of objects in space, but it can give us a lot of information about Earth as well, including new Physics, information about how the Earth was formed, and even help us locate ourselves in time and space!  Astronomy 101 is the introductory Astronomy course, complemented by its lab section Astronomy 101L.  Please note I reserve the right to change this syllabus if I see it fit to do so throughout the semester.  \b{(pre- or co-requisite: None)}

\section*{Textbook and tools}
The textbook for the course, \i{Astronomy} from OpenStax, is free!  However, it's also crucial to the course, so be sure to download it from \url{https://openstax.org/details/books/astronomy} with time to spare.  Regular weekly readings and homework will be assigned from the textbook, and homework, quizzes, and tests will derive from text and lecture material.

Since this is an online course, homework will be completed on Word and submitted through UNM Learn.  A free alternative to Word that you can download is \href{https://www.libreoffice.org/download/download/}{Libreoffice}, available at \url{https://www.libreoffice.org/download/download/}.  You are also expected to have access to high-speed internet and a computer.  You can do this at the computer labs on campus or a public library if you prefer.  Finally, be sure you can open pdf files; if you can't open the homework files on \url{http://helasraizam.github.io}, you can download \href{https://get.adobe.com/reader/}{Adobe Reader} for free at \url{https://get.adobe.com/reader/}.

\section*{Objectives}
Topics to look forward to include a brief history of Astronomy, an analysis of the behavior of stars and planets as seen from Earth, applying the scientific method, understanding the scales of the universe, how to use tools like telescopes and spectroscopes to observe and quantify the stars, a study of the formation and properties of objects in our solar system, na overview of gravity and electromagnetism, methods of discovery of planets around stars, the structure and activity of the Sun and its contextualization with other stars, the life cycle of a star, the structure of the Milky way and its comparison to other galaxies, the Big Bang theory in the context of recent observations, and the possibility of extraterrestrial life in the universe.

\section*{Grading}
The grading is outlined below, with grade percentages on the border resulting in the higher grade:\\
\begin{minipage}[t]{.5\textwidth}\centering
    \begin{tabular}{l l}
      Homework&40\%\\
      Projects&40\%\\
      Quizzes&20\%\\
    \end{tabular}
\end{minipage}
\begin{minipage}[t]{.5\textwidth}\centering
    \begin{tabular}{l l}
      99-100&A+\\
      94-99&A\\
      90-94&A-\\
      87-90&B+\\
      84-87&B\\
      80-84&B-\\
      70-80&C\\
      60-70&D\\
      0-60&F
    \end{tabular}
\end{minipage}
Please see \url{http://helasraizam.github.io} for a table of due dates.  Every week, you're expected to keep up with the recorded lecture(s), review the course notes, and complete the weekly homework and quiz on \href{http://learn.unm.edu}{UNM Learn}.  If you feel you are falling behind, contact me as soon as possible.  \b{All work is due at 11:59 pm at the date posted on UNM Learn, \url{http://helasraizam.github.io}}, and you should expect your work to be graded within two weeks after its deadline.  Feedback will be on the UNM Learn submission, in comment boxes.  You must submit a word (.doc) or pdf (.pdf) file.

You must show your work to get full credit and good feedback (I can’t give you partial credit if I don’t know why your answer is wrong!); you can take pictures of your work and copy/paste it into the word document.  All assigned work will include a point layout describing how many points each problem and sub-problem is worth---if parts of a problem don’t have points assigned, you should assume the points for that problem are evenly distributed.  On rare occasions I will reassign the points based on student performance (e.g., if a question has a typo)---these reassignments will be to your benefit.
\subsubsection*{Quizzes (20\%)}
The quizzes are basic questions to assess your understanding of that lecture.  You can find the quizzes on UNM Learn.
\subsubsection*{Homework (40\%)}
The homework is designed to extend your understanding.  Homework will be typed on your computer and submitted on UNM Learn.  You can use Word, or you can download \href{https://www.libreoffice.org/download/download/}{Libreoffice}, a free alternative, at \url{https://www.libreoffice.org/download/download/}.  You are encouraged to attend office hours and discuss the topics of the homework on the UNM Learn forums (without posting the answers).
\subsubsection*{Projects (40\%)}
This course will include three projects, which summarize the set of chapters associated with that project.  You should plan to type up the projects.  You should leave plenty of time for the projects.
\newpage

\section*{Tutoring and Support}
You are encouraged to send me and your classmates your questions on the topics covered in the course on the UNM Learn discussion page for the appropriate chapter(s); this helps your other classmates who may have the same questions.  You should expect a response within 24 hours of a post.  Please don't make the mistake of thinking you're the only one not to understand a topic---if you don't understand a topic, chances are your peers don't either, and bringing it up will remind me to cover it during office hours or the next lecture.

In-person tutoring is also available at the \href{http://valencia.unm.edu/campus-resources/the-learning-center/index.html}{Learning Center}, see \url{http://valencia.unm.edu/campus-resources/the-learning-center/index.html}.  If you're spending more than four hours on the homeworks, please let me know.

\section*{Online Course}
This is an online course, meaning you'll have to schedule the time during the week to watch the lectures, complete the quizzes, assignments, and projects on your own.  You should do your best not to fall behind as it will be detrimental to your grade.  You should plan to spend 6-9 hours a week for this class, which includes new topics introduced in lectures.

\section*{Students with Disabilities}
Qualified students with disabilities needing appropriate academic adjustments should contact me as soon as possible to ensure your needs are met.

\section*{Title IX}
UNM faculty are considered ``reponsible employees'' by the Department of Education.\footnote{See p. 15 - \url{http://www2.ed.gov/about/offices/list/ocr/docs/qa-201404-title-ix.pdf}}  This designation requires that any report of gender discrimination which includes sexual harassment, sexual misconduct, and sexual violence made to a faculty member, TA, or GA must be reported to the Title IX Coordinator at the Office of Equal Opportunity (\url{oeo.unm.edu}).  For more information on the campus policy regarding sexual misconduct, see: \url{https://policy.unm.edu/university-policies/2000/2740.html}.

\end{document}
%\bibliographystyle{plain}
%\bibliography{References}
%\printbibliography[title=Bibliography]

%%% Local Variables:
%%% LaTeX-command: "latex -shell-escape"
%%% End:
