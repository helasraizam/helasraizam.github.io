\documentclass[12pt]{article}

\usepackage{Physics,python,miller,indentfirst}
% style=verbose-trad1
% style=authortitle-icomp [default]
\usepackage[style=verbose-trad1]{biblatex} 
\pagestyle{fancy}
\pagenumbering{gobble}
\addbibresource{References.bib}

\begin{document}

\lhead{ASTR 101L}
\rhead{Spring 2019}
%\chead{Maziar Saleh Ziabari}

\newcommand{\TODO}[1]{}

\br\vspace{0cm}\begin{center}\Huge Astronomy 101 Lab Syllabus\end{center}\vspace{.5cm}

\noindent\begin{tabular}{>{\it}r l}
  Instructor:&Maziar Saleh Ziabari\\
  Email:&helasraizam@unm.edu\\
  OH:&TBD\\
  Classroom:&online\\
  Site:&\url{https://learn.unm.edu}, \url{https://helasraizam.github.io}
\end{tabular}
\vspace{.7cm}

\noindent Welcome to Astronomy 101L!  This lab section complements the Astronomy 101 lectures by building your intuition through experiments and theory.  As such, please do not hesitate to ask any questions you have on the material; the ultimate aim is for you to understand concepts and be proficient in introductory Astronomy.  Please note I reserve the right to change this syllabus if I see it fit to do so throughout the semester.  \b{(pre- or co-requisite: ASTR 101)}

\section*{Objectives}
Topics to look forward to include a brief history of Astronomy, an analysis of the behavior of stars and planets as seen from Earth, applying the scientific method, understanding the scales of the universe, how to use tools like telescopes and spectroscopes to observe and quantify the stars, a study of the formation and properties of objects in our solar system, na overview of gravity and electromagnetism, methods of discovery of planets around stars, the structure and activity of the Sun and its contextualization with other stars, the life cycle of a star, the structure of the Milky way and its comparison to other galaxies, the Big Bang theory in the context of recent observations, and the possibility of extraterrestrial life in the universe.  Note that the lab is a math-based course, and the math is more math-intensive than the lecture; a knowledge of geometry and algebra is expected.

\section*{Title IX}
UNM faculty, Teaching Assistants, and Graduate Assistants are considered ``reponsible employees'' by the Department of Education.\footnote{See p. 15 - \url{http://www2.ed.gov/about/offices/list/ocr/docs/qa-201404-title-ix.pdf}}  This designation requires that any report of gender discrimination which includes sexual harassment, sexual misconduct, and sexual violence made to a faculty member, TA, or GA must be reported to the Title IX Coordinator at the Office of Equal Opportunity (\url{oeo.unm.edu}).  For more information on the campus policy regarding sexual misconduct, see: \url{https://policy.unm.edu/university-policies/2000/2740.html}.


\section*{Tutoring and Support}
You are encouraged to send me and your classmates your questions on the topics covered in the course on the UNM Learn discussion page for the appropriate chapter(s); this helps your other classmates who may have the same questions.  Please don't make the mistake of thinking you're the only one not to understand a topic---if you don't understand a topic, chances are your peers don't either, and bringing it up will remind me to cover it during office hours or the next lecture.  You can access the Astronomy 101 course website, which is the same URL as the course website for the lab, and watch videos to refresh concepts for labs whether or not you're enrolled in the course.

In-person tutoring is also available at the \href{http://valencia.unm.edu/campus-resources/the-learning-center/index.html}{Learning Center}, see \url{http://valencia.unm.edu/campus-resources/the-learning-center/index.html}.  If you're spending more than four hours on the labs, please let me know.

\section*{Online Course}
This is an online course, meaning you'll have to schedule the time during the week to watch the lectures, complete the quizzes, assignments, and projects on your own.  You should do your best not to fall behind as it will be detrimental to your grade.  You should plan to spend 2-4 hours a week for this class.


\section*{Grading}
Your course grade is determined by your grades in the labs and the three naked-eye observing projects.  Each of the 15 labs is 100 points, the observing projects are each worth 100 points, and there is one homework assignment worth 50 points, for a total of 1850 points in the course.  The labs will get more challenging as the course goes on, so don't skip any labs.  Labs are available on UNM Learn, you can also consult \url{http://helasraizam.github.io} for tentative deadlines.  The grading scale is included below:

\begin{center}\begin{tabular}{l l@{\hskip .5in}l l@{\hskip .5in}l l}
  99-100&A+ &87-90&B+ &70-80&C\\
  94-99&A &84-87&B &60-70&D\\
  90-94&A- &80-84&B- &0-60&F
\end{tabular}\end{center}

\section*{Students with Disabilities}
Qualified students with disabilities needing appropriate academic adjustments should contact me as soon as possible to ensure your needs ar emet.  Handouts are available in alternative accessible formats upon request.


\end{document}
%\bibliographystyle{plain}
%\bibliography{References}
%\printbibliography[title=Bibliography]

%%% Local Variables:
%%% LaTeX-command: "latex -shell-escape"
%%% End:
